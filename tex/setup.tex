% IMPORTANTE
% Para poner barrabaja en una url hay que cambiar '_' por '\textunderscore'

%========================================================================================
% PAQUETES
%=======================================================================================
\usepackage[table]{xcolor}
\usepackage[spanish]{babel} % Indicar idioma español
\usepackage[utf8]{inputenc} % Poder poner tildes directamente
\usepackage{graphicx} % Para incluir gráficos
\usepackage{cite} % Para crear referencias
\usepackage{svg} % Para usar imágenes en formato vectorial
\usepackage{float} % Para modificar como aparecen las imagenes
\usepackage{listings} % Para introducir código
\usepackage{times}
\usepackage{color} % para añadir colores
\usepackage[pdftex,
			pdfauthor={Juan Ignacio Cuerno Lagos},
			pdftitle={Extensión móvil para sistemas de ventas propietarios(Intraza)},
			pdfsubject={Intraza},
			pdfkeywords={Android, Java, Grails},
			pdfproducer={Latex},
			pdfcreator={pdflatex}]{hyperref}  % Para que se creen enlaces en el documento (a otras partes o a la web) y para añadir metadatos al PDF generado
\usepackage{AnonymousPro}
\usepackage{url}
\usepackage[Bjornstrup]{fncychap} % Para poner los títulos de los capítulos más bonitos
\usepackage{fancyhdr} % Para configurar las cabeceras y los piés de página
\usepackage{titling} % Para usar la información del documento en otras partes de él
\usepackage{caption}
\usepackage{minted}
\usepackage{pgfplots}
\usepackage{pgfplotstable}
\usepackage{tikz}
\usepackage{eurosym}
\usepackage{wrapfig}
\usepackage{blindtext}
\usepackage{nameref}
\usepackage{xcolor}
\usepackage{amstext} % for \text
\DeclareRobustCommand{\officialeuro}{%
\ifmmode\expandafter\text\fi
{\fontencoding{U}\fontfamily{eurosym}\selectfont e}}


%========================================================================================
% ESTILO
%========================================================================================

%----------------------------------------------------------------------------------------
% RUTAS PARA LAS IMAGENES
%----------------------------------------------------------------------------------------

\graphicspath{{./figuras/}}

%----------------------------------------------------------------------------------------
% ARREGLO PARA EL INDICE DE FIGURAS
%----------------------------------------------------------------------------------------

\makeatletter
\renewcommand*\l@figure{\@dottedtocline{1}{1em}{3.2em}}
\makeatother

%----------------------------------------------------------------------------------------
% TIPOGRAFIA
%----------------------------------------------------------------------------------------

\renewcommand{\familydefault}{\sfdefault} % Cambia la tipografía de todo el documento
\setlength{\parskip}{2mm} % Define la distancia entre parrafos (por defecto = 0)

%----------------------------------------------------------------------------------------
% CABECERAS Y PIES DE PAGINA
%----------------------------------------------------------------------------------------
\pagestyle{fancy}
\fancyhf{}
\fancyhead[CE,CO]{\leftmark}
\fancyhead[LE,RO]{\thepage}
\fancyfoot[RE,LO]{Juan Ignacio Cuerno Lagos}
\fancyfoot[LE,RO]{\thepage}

\renewcommand{\headrulewidth}{2pt}
\renewcommand{\footrulewidth}{1pt}

%----------------------------------------------------------------------------------------
% ENLACES
%----------------------------------------------------------------------------------------

\hypersetup{
	colorlinks,
	citecolor=black,
	filecolor=black,
	linkcolor=black,
	urlcolor=blue
}

%----------------------------------------------------------------------------------------
% NUMERO DE PROFUNDIDAD DE NUMERACION DE CONENIDOS
%----------------------------------------------------------------------------------------
\setcounter{tocdepth}{3}
\setcounter{secnumdepth}{3}

%----------------------------------------------------------------------------------------
% CODIGO
%----------------------------------------------------------------------------------------

% Parte para los comandos (con lstlisting)
\definecolor{gray95}{gray}{.95}
\definecolor{gray85}{gray}{.80}
\definecolor{gray45}{gray}{.45}
\definecolor{mypink}{RGB}{177,48,112}
\definecolor{myblue}{RGB}{56,133,231}
\definecolor{shadecolor}{rgb}{0.61, 0.77, 0.89}
\definecolor{HeaderCol}{rgb}{0.30, 0.6, 0.8}
\definecolor{tittletable}{rgb}{0.13, 0.7, 0.8}
\definecolor{carolinablue}{rgb}{0.6, 0.73, 0.89}
\definecolor{dkgreen}{rgb}{0,0.6,0}
\definecolor{gray}{rgb}{0.5,0.5,0.5}
\definecolor{mauve}{rgb}{0.58,0,0.82}

\lstset{ 
	frame=Ltb,
	framerule=0pt,
	aboveskip=0.5cm,
	framextopmargin=3pt,
	framexbottommargin=3pt,
	framexleftmargin=0.2cm,
	framesep=0pt,
	rulesep=2.0pt,
	backgroundcolor=\color{gray95},
	rulesepcolor=\color{black},
	%
	stringstyle=\ttfamily,
	showstringspaces = false,
	basicstyle=\small\ttfamily,
	commentstyle=\color{gray45},
	keywordstyle=\bfseries,
	%
	numbers=none,
	breaklines=true,
}

% Minimizar fragmentado de listados
%\lstnewenvironment{listing}[1][]
%{\lstset{#1}\pagebreak[0]}{\pagebreak[0]}

% Estilo de codigo para la consola
\lstdefinelanguage{none}{}
\lstdefinestyle{consola}{
	language=none,
	breaklines=true,
	basicstyle=\footnotesize\bf\ttfamily,
	backgroundcolor=\color{gray85},
	rulesepcolor=\color{carolinablue},
	numbers=none,
}

\lstdefinestyle{JAVA}{
language=Java,
aboveskip=3mm,
belowskip=3mm,
showstringspaces=false,
columns=flexible,
basicstyle={\small\ttfamily},
numbers=none,
numberstyle=\tiny\color{gray},
keywordstyle=\color{blue},
commentstyle=\color{dkgreen},
stringstyle=\color{mauve},
breaklines=true,
breakatwhitespace=true,
tabsize=3
}


% Parte para el código (con minted) (minted es mejor para código que lstlisting)
\usemintedstyle{Java}

\fboxsep=5mm%padding
\fboxrule=2pt%border}

