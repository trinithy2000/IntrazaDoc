\chapter{Conclusiones}


\section{Introducción}

Durante la realización del proyecto se procede a la ejecución de las diferentes fases.  Analizando la viabilidad, planificando a traves de metodologías ágiles SCRUM, se describen  y enumeran las tecnologías y fundamentos teóricos. Se analiza la estructura que evalúa los resultados. En los anexos se definen y aclaran conceptos, además de un manual de usuario y diferentes guías  de instalación y uso de los diferentes componentes. 

\section{Conclusiones}

A continuación vamos a recordar cuál era el Objetivo principal del proyecto marcado al inicio del mismo: 

Construir una aplicación en Android que permita operar con el gestor de pedidos del ERP de la compañía de manera remota. A la finalización del proyecto, se puede afirmar que se ha conseguido satisfacer completamente el Objetivo principal. El grado de consecución se puede considerar bastante alto, resaltando el hecho de que la empresa empezó a operar con la app en la campaña de verano, sin incidentes reseñables. 

El sistema/módulo desarrollado cumple perfectamente el objetivo propuesto y además está totalmente preparado para realizar futuras ampliaciones e integraciones. 

La consecución del Objetivo principal incluía la satisfacción de unos sub-objetivos que van totalmente ligados. A continuación se describen los que se han satisfecho: 

- El principal objetivo a cumplir a nivel personal fue adquirir conocimientos sobre el entorno y el contexto del proyecto.

Siendo estos variados como:
\begin{itemize}
	\item \emph{\textbf{Conocer las principales características del lenguaje Java:}} Sin tener un previo conocimiento básico del lenguaje Java habría sido imposible programar la aplicación en Android. 

	\item \emph{\textbf{Conocer las principales características de Android:}} Al igual que en el punto anterior, si no se hubiera estudiado y entendido las principales características de Android, así como su funcionamiento, llevar a cabo este proyecto con el desarrollo de la aplicación habría sido imposible. 

	\item \emph{\textbf{Estudiar el entorno de desarrollo de Android:}} Con la elección del entorno Android studio, fue necesario estudiarlo, entenderlo y manipularlo con la suficiente soltura que nos permitiera el correcto desarrollo de la aplicación. 

	\item \emph{\textbf{Conocer funcionamiento y desarrollo de aplicaciones J2EE:}}
Sin este conocimiento, habría sido imposible desarrollar el servicio web de conexión.

	\item \emph{\textbf{Conocer funcionamiento y manejo de bases de datos:}} Sin este conocimiento, no podríamos abordar la persistencia de datos solicitada. 

	\item \emph{\textbf{Conocer funcionamiento y manejo de servidores de aplicaciones java:}} Era necesario conocer los diferentes entornos de hospedaje de aplicaciones, ya sea tomcat, jboss, websphere, etc. 
	
\end{itemize}

Otro de los objetivos a cumplir en este proyecto era la realización de la memoria de proyecto debidamente cumplimentada con el reglamento existente. 

\section{Planificación temporal final}

Por otro lado estaba el factor temporal. El proyecto se debía ajustar a una planificación realizada al comienzo del mismo. La duración era de 3 meses aproximadamente. Ya que se quería tener operativa la aplicación antes de la campaña de verano, de vital importancia para la empresa. 

Pero por lo general, a la finalización del proyecto, se puede afirmar que se ha logrado seguir bastante bien la planificación temporal realizada en su comienzo, siempre con las pequeñas y lógicas desviaciones. 

\section{Conclusión final}

Con lo visto anteriormente, podemos afirmar que, a la finalización del proyecto, el grado de consecución de objetivos es bastante alto y que se ha logrado ajustar bastante bien su desarrollo en el espacio temporal planificado. Por tanto, se puede concluir que el proyecto ha finalizado con éxito. 



